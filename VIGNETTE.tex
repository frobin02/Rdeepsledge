% Options for packages loaded elsewhere
\PassOptionsToPackage{unicode}{hyperref}
\PassOptionsToPackage{hyphens}{url}
%
\documentclass[
]{article}
\usepackage{amsmath,amssymb}
\usepackage{iftex}
\ifPDFTeX
  \usepackage[T1]{fontenc}
  \usepackage[utf8]{inputenc}
  \usepackage{textcomp} % provide euro and other symbols
\else % if luatex or xetex
  \usepackage{unicode-math} % this also loads fontspec
  \defaultfontfeatures{Scale=MatchLowercase}
  \defaultfontfeatures[\rmfamily]{Ligatures=TeX,Scale=1}
\fi
\usepackage{lmodern}
\ifPDFTeX\else
  % xetex/luatex font selection
\fi
% Use upquote if available, for straight quotes in verbatim environments
\IfFileExists{upquote.sty}{\usepackage{upquote}}{}
\IfFileExists{microtype.sty}{% use microtype if available
  \usepackage[]{microtype}
  \UseMicrotypeSet[protrusion]{basicmath} % disable protrusion for tt fonts
}{}
\makeatletter
\@ifundefined{KOMAClassName}{% if non-KOMA class
  \IfFileExists{parskip.sty}{%
    \usepackage{parskip}
  }{% else
    \setlength{\parindent}{0pt}
    \setlength{\parskip}{6pt plus 2pt minus 1pt}}
}{% if KOMA class
  \KOMAoptions{parskip=half}}
\makeatother
\usepackage{xcolor}
\usepackage[margin=1in]{geometry}
\usepackage{color}
\usepackage{fancyvrb}
\newcommand{\VerbBar}{|}
\newcommand{\VERB}{\Verb[commandchars=\\\{\}]}
\DefineVerbatimEnvironment{Highlighting}{Verbatim}{commandchars=\\\{\}}
% Add ',fontsize=\small' for more characters per line
\usepackage{framed}
\definecolor{shadecolor}{RGB}{248,248,248}
\newenvironment{Shaded}{\begin{snugshade}}{\end{snugshade}}
\newcommand{\AlertTok}[1]{\textcolor[rgb]{0.94,0.16,0.16}{#1}}
\newcommand{\AnnotationTok}[1]{\textcolor[rgb]{0.56,0.35,0.01}{\textbf{\textit{#1}}}}
\newcommand{\AttributeTok}[1]{\textcolor[rgb]{0.13,0.29,0.53}{#1}}
\newcommand{\BaseNTok}[1]{\textcolor[rgb]{0.00,0.00,0.81}{#1}}
\newcommand{\BuiltInTok}[1]{#1}
\newcommand{\CharTok}[1]{\textcolor[rgb]{0.31,0.60,0.02}{#1}}
\newcommand{\CommentTok}[1]{\textcolor[rgb]{0.56,0.35,0.01}{\textit{#1}}}
\newcommand{\CommentVarTok}[1]{\textcolor[rgb]{0.56,0.35,0.01}{\textbf{\textit{#1}}}}
\newcommand{\ConstantTok}[1]{\textcolor[rgb]{0.56,0.35,0.01}{#1}}
\newcommand{\ControlFlowTok}[1]{\textcolor[rgb]{0.13,0.29,0.53}{\textbf{#1}}}
\newcommand{\DataTypeTok}[1]{\textcolor[rgb]{0.13,0.29,0.53}{#1}}
\newcommand{\DecValTok}[1]{\textcolor[rgb]{0.00,0.00,0.81}{#1}}
\newcommand{\DocumentationTok}[1]{\textcolor[rgb]{0.56,0.35,0.01}{\textbf{\textit{#1}}}}
\newcommand{\ErrorTok}[1]{\textcolor[rgb]{0.64,0.00,0.00}{\textbf{#1}}}
\newcommand{\ExtensionTok}[1]{#1}
\newcommand{\FloatTok}[1]{\textcolor[rgb]{0.00,0.00,0.81}{#1}}
\newcommand{\FunctionTok}[1]{\textcolor[rgb]{0.13,0.29,0.53}{\textbf{#1}}}
\newcommand{\ImportTok}[1]{#1}
\newcommand{\InformationTok}[1]{\textcolor[rgb]{0.56,0.35,0.01}{\textbf{\textit{#1}}}}
\newcommand{\KeywordTok}[1]{\textcolor[rgb]{0.13,0.29,0.53}{\textbf{#1}}}
\newcommand{\NormalTok}[1]{#1}
\newcommand{\OperatorTok}[1]{\textcolor[rgb]{0.81,0.36,0.00}{\textbf{#1}}}
\newcommand{\OtherTok}[1]{\textcolor[rgb]{0.56,0.35,0.01}{#1}}
\newcommand{\PreprocessorTok}[1]{\textcolor[rgb]{0.56,0.35,0.01}{\textit{#1}}}
\newcommand{\RegionMarkerTok}[1]{#1}
\newcommand{\SpecialCharTok}[1]{\textcolor[rgb]{0.81,0.36,0.00}{\textbf{#1}}}
\newcommand{\SpecialStringTok}[1]{\textcolor[rgb]{0.31,0.60,0.02}{#1}}
\newcommand{\StringTok}[1]{\textcolor[rgb]{0.31,0.60,0.02}{#1}}
\newcommand{\VariableTok}[1]{\textcolor[rgb]{0.00,0.00,0.00}{#1}}
\newcommand{\VerbatimStringTok}[1]{\textcolor[rgb]{0.31,0.60,0.02}{#1}}
\newcommand{\WarningTok}[1]{\textcolor[rgb]{0.56,0.35,0.01}{\textbf{\textit{#1}}}}
\usepackage{graphicx}
\makeatletter
\def\maxwidth{\ifdim\Gin@nat@width>\linewidth\linewidth\else\Gin@nat@width\fi}
\def\maxheight{\ifdim\Gin@nat@height>\textheight\textheight\else\Gin@nat@height\fi}
\makeatother
% Scale images if necessary, so that they will not overflow the page
% margins by default, and it is still possible to overwrite the defaults
% using explicit options in \includegraphics[width, height, ...]{}
\setkeys{Gin}{width=\maxwidth,height=\maxheight,keepaspectratio}
% Set default figure placement to htbp
\makeatletter
\def\fps@figure{htbp}
\makeatother
\setlength{\emergencystretch}{3em} % prevent overfull lines
\providecommand{\tightlist}{%
  \setlength{\itemsep}{0pt}\setlength{\parskip}{0pt}}
\setcounter{secnumdepth}{-\maxdimen} % remove section numbering
\usepackage{booktabs}
\usepackage{longtable}
\usepackage{array}
\usepackage{multirow}
\usepackage{wrapfig}
\usepackage{float}
\usepackage{colortbl}
\usepackage{pdflscape}
\usepackage{tabu}
\usepackage{threeparttable}
\usepackage{threeparttablex}
\usepackage[normalem]{ulem}
\usepackage{makecell}
\usepackage{xcolor}
\ifLuaTeX
  \usepackage{selnolig}  % disable illegal ligatures
\fi
\IfFileExists{bookmark.sty}{\usepackage{bookmark}}{\usepackage{hyperref}}
\IfFileExists{xurl.sty}{\usepackage{xurl}}{} % add URL line breaks if available
\urlstyle{same}
\hypersetup{
  pdftitle={package Rdeepsledge},
  pdfauthor={FROBIN02},
  hidelinks,
  pdfcreator={LaTeX via pandoc}}

\title{package Rdeepsledge}
\author{FROBIN02}
\date{}

\begin{document}
\maketitle

\hypertarget{introduction}{%
\section{Introduction}\label{introduction}}

The RDeepSledge package is a toolbox for underwater surveys using a
benthic sledge. Sledge location (longitude and latitude) of the benthic
sledge can be estimate from boat position with
\textbf{DeepS\_location()}. Twins green laser positions can be extract
from movie using \textbf{DeepS\_find\_laser()} and correction of laser
positions can be done with \textbf{DeepS\_correct\_laser()}.

\hypertarget{installation}{%
\section{Installation}\label{installation}}

To install MonPackage, use
\textbf{devtools::install\_github(``FROBIN02/Rdeepsledge'')}.

\hypertarget{estimating-sledge-location-from-boat-position}{%
\subsection{Estimating Sledge location from boat
position}\label{estimating-sledge-location-from-boat-position}}

\emph{in process}

\hypertarget{extraction-of-laser-point-locations}{%
\subsection{Extraction of laser point
locations}\label{extraction-of-laser-point-locations}}

Note: the AV package must be installed separately.

\begin{Shaded}
\begin{Highlighting}[]
\FunctionTok{library}\NormalTok{(Rdeepsledge)}

\NormalTok{MP4\_file }\OtherTok{\textless{}{-}} \FunctionTok{file.choose}\NormalTok{ ()}

\NormalTok{RESULT}\OtherTok{\textless{}{-}} \FunctionTok{DeepS\_find\_laser}\NormalTok{(MP4\_file,}\ConstantTok{FALSE}\NormalTok{,}\DecValTok{120}\NormalTok{)}
\end{Highlighting}
\end{Shaded}

Here, \emph{MP4\_file} is your video file, \emph{FALSE} is the
parallelization option (not stable for this version), and the number
\emph{120} corresponds to the number of frames between each estimation.
This value cannot be less than 40.

\textbf{WARNING} This function may run for several days. The conversion
of the video into frames can take up space, but the temporary folder is
deleted at the end of the execution.

\hypertarget{example-for-correcting-laser-position-files}{%
\subsection{Example for correcting laser position
files}\label{example-for-correcting-laser-position-files}}

\begin{Shaded}
\begin{Highlighting}[]
\FunctionTok{library}\NormalTok{(Rdeepsledge)}

\NormalTok{RESULT }\OtherTok{\textless{}{-}} \FunctionTok{read.table}\NormalTok{(}\StringTok{"laser\_position\#yourvideo\#.txt"}\NormalTok{) }\CommentTok{\# in "USER/document"}
\end{Highlighting}
\end{Shaded}

The file extracted from the video is stored under the name RESULT and a
backup is made in USER/document in the form
laser\_position\emph{VOTREVIDEO}.txt.

For the example, the data MPO\_REFURE\_source.R replaces the data file.

\begin{Shaded}
\begin{Highlighting}[]
\FunctionTok{library}\NormalTok{(Rdeepsledge)}

\NormalTok{mylaser }\OtherTok{\textless{}{-}} \FunctionTok{DeepS\_correct\_laser}\NormalTok{(RESULT)}
\end{Highlighting}
\end{Shaded}

exemple de données

\begin{table}

\caption{\label{tab:cor res}Table 1:tableau de données avec balise}
\centering
\begin{tabu} to \linewidth {}
\hline

\hline
\end{tabu}
\end{table}

The details of the fields are not yet available for this version\ldots{}

Next, a figure extracted from the corrected data.

NULL

\end{document}
